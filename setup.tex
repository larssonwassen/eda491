\section{System Configuration and Requirements}
\label{sec:setup}

At the beginning of the \lab{} an initial firewall configuration was in place to make the hosts basic functionality work as intended. The configuration can be seen in Listing~\ref{lst:fw-init}, which was applied using the script shown in Appendix~\ref{app:fw-init}. 

The firewall had four so called chains that were monitored. The INPUT chain consisted of rules concerning all the incoming traffic from the network interface, the OUTPUT chain held rules for all the outgoing traffic from the host and the FORWARD chain held rules for when the host should route traffic. The FORWARD chain could for example be used to implement a router.

The rules that were present in the initial firewall configuration were mainly for the file sharing protocol NFS that was used by all client hosts on the Chalmers network. Without these rules the hosts would not be able to communicate with the NFS server and thereby not be able to access files that are crucial for the system to run properly. The other rules that were present, more specifically line number 2 and 3 in the INPUT chain and line number 1 and 2 in the FORWARD chain were there to drop invalid TCP packets. These rules protect from for example so called Xmas and TCP NULL scans and from other undefined behaviour when receiving invalid TCP packets.

For the \lab{} there were some requirements on functionality that had to be met. Except for the previous mentioned rules for NFS rules for the services SSH, portmapper, and a webserver on ports 22, 111, and 8080 respectively were to created. Aaother requirement was also that the host should still be able to surf the web and use other common services. 

One thing to note about the default configuration is that the default policy for all the four chains were set to ACCEPT. The choice that exist for this is either DROP which, as its name implies, drops all traffic that is not permitted by the rules or ACCEPT which accepts all traffic that is not denied by the rules. The latter choice opens up for security risks in a higher manner if the system is not properly configured than the first choice does. For the policy DROP there is a higher risk that the firewall drops valid data than it is to accept malicious data, but for policy ACCEPT it is the other way around. Generally from a security point of view we would say that a default policy of ACCEPT is a bad way to setup a firewall. Also, as a firewall with a default policy of ACCEPT will probably require many more rules than a firewall with default policy DROP it will probably need faster hardware for handling the same load.

\newpage
\lstinputlisting[caption=Initial firewall configuration,label=lst:fw-init]{firewall-init.txt}