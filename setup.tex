\section{System Configuration and Requirements}
\label{sec:setup}

At the begining of the \lab a initial firewall configuration was in place to make the hosts basic functionallity work as intended. The configuration can be seen in Listing~\ref{lst:fw-init} which were applied applied using the script shown in Appendix~\ref{app:fw-init}. 

The firewall had four so called chains that were monitored. The INPUT chain consisted of rules concerning all the incomming traffic from the network interface, the OUTPUT chain held rules for all the outgoining traffic from the host and the FORWARD chain hold rules for when the host should route traffic. This could for example be used to implement a router.

The rules that were present were mainly for the file sharing protocol NFS which was used by all client hosts on the chalmers network. Without these rules the hosts would not be able to communicate with the NFS server and thereby not be able to access files that are crucial for the system to run properly. The other rules that were present, more specifically line number 2 and 3 in the INPUT chain and line number 1 and 2 in the FORWARD chain were there to drop invalid TCP packets. This were done to protect from for example so called Xmas and TCP NULL scans and from other undefined behaviour when receiving invalid TCP packets.

One thing to note about the default configuration is that a thing called defualt policy for all the different chains were set to ACCEPT. The choice you have for this is either DROP which, as its name implies, drops all traffic that is not permitted by the rules or ACCEPT which accepts all traffic that is not denied by the rules. The latter choice opens up for security risks in a higher maner if the system is not properly configured than the first choice does. If the choice was to drop all the unspecified traffic the risk is heigher that the firewall drops valid data than to it is to accept malicious data, but if the firewall accept all traffic that is not blocked by rules it is more likely that it actually accepts malicious data. Generally from a security point of view we would say that a default policy of ACCEPT is a bad way to setup a firewall. Also, as you will have to have rules for all packets that should be droped, which probably is many more than should be accepted, a system under heavy load will work slower with a default policy of ACCEPT than with DROP.


\lstinputlisting[caption=Initial firewall configuration,label=lst:fw-init]{firewall-init.txt}

