\section{Firewall Configuration}
\label{sec:config}

The final firewall configuration is shown in listening \ref{list:final_firewall_config} and displays all the firewall's rules. This inital firewall configuration in section \ref{sec:setup} is a subset of the final firewall configuration due to that the intial configuration has rules that are needed to operate the computer. We added a number of rules that are presented here below but we also changed the default policy from ACCEPT to DROP for all the three chains INPUT, OUTPUT and FORWARD. The default policy DROP is more secure as explained in \ref{sec:setup}. The chains of interest in this \lab was INPUT and OUTPUT and the other chains are not considered or tested from a security perspective. To configure the firewall, the script in Appendix~\ref{app:fw-final} was issued.

The added rules maked certain unwanted traffic were droped and valid traffic accepted. First, all the traffic where checked for malicious or malformed content and droped if they matched any of those rules. The denying rules covered IP spoofing and ping flooding. Second followed a number of rules that checked whether the content seemd valid and if a packed matched any of these rules, it was accepted and forwarded. The allowing rules let through limited ping requests, established TCP connections, traffic sent in loopback mode, outgoing traffic and new connections to specified services. Lastly was the traffic that had not been matched by either the denying or the allowing rules logged before the default action - drop - was issued. The rules with their purpose and the mentioned organization can easily be seen in the script in Appendix~\ref{app:fw-final} and in the continued text the rules will be refered to where in listening \ref{list:final_firewall_config} they can be found.

The first problem was concerning IP spoofing. It is not possible to ensure that an IP is not spoofed, so there is no way to completely overcome this problem. What usally is done - and what we also did - was to drop all incoming packages with a source IP address that did not belong to the current network, which correspond to the rules on lines 4-7. Further was rules from line 31 to 46 added to drop all the outgoing traffic that either had a source or a destination IP address that did not belong to the current network. 

The second problem was ping flooding. A few ping messages are normal to receive and also important to answer by transmitting an echo, but ping request can also be used as an DOS attack and for that reason a defined number of echo requests per second can be reasonable to accept into the system. The rule on line 10 restrict the number of accepted echo requests to one per second and it will be applied after a burst of five answered echo requests. To drop the remaining echo requests that was not accepted by the restriction rule, a rule on line 11 was added to drop all echo requests.

The first allowing rules accepts established TCP connections, the second traffic sent in loopback mode and the third outgoing traffic. This rule is on line 12 and accepts all TCP connections with the states either related or established.


The rules seen in \ref{list:final_firewall_config} that has not been addressed by the above text are the rules from the inital firewall configuration, which are described in section \ref{sec:setup}.

\inlinetodo{Describe the new firewall configuration, together with the output, e.g., 
rule on line 5 ensures that the number of ping packets are limited to 1. Don't 
forget to refer to your script in Appendix~\ref{app:fw-final}.}

\lstinputlisting[caption=Final firewall 
configuration,label=lst:fw-config]{firewall-final.txt}\label{list:final_firewall_config}



