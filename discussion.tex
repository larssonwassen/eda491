\section{Discussion} 
\label{sec:discussion}

The firewall configuration made is not wisely to actually use. It was made purely to give insight in how to configure this particular type of firewall while allowing access to the particular services specified in this report. On the other hand, the configuration is a good start towards a complete configuration.

The testing did not cover the defined firewall rules regarding IP spoofing and malformed TCP headers. They were not tested due to that we followed the \lab{} PM and only run the \texttt{nmap} command to investigate the states of all the ports on the host. Unfortunately, we did not really look through that we had verified all the rules, only that we got the desired output from \texttt{nmap}. If we would have done the lab again the omitted rules would have been tested through other \texttt{nmap} commands. 

If we did test the IP spoofing we would probably see that the defined rules, seen in Listening \ref{list:final_firewall_config}, were not that good. A later inspection showed that the IP spoofing rules should have been defined in another manner. For the OUTPUT chain the rules on line 31 to 46 are unfortunately designed in such a manner that both the source and the destination addresses has to be invalid in order to be dropped, rather than - as it should be - if only one of the addresses were invalid. In the INPUT chain the rules regarding IP spoofing should be redefined as well. As the rules are defined now, the host will accept traffic with any destination address. To only receive traffic addressed to the host, a rule should be inserted on line 2 two drop packages not destined to the host's IP address.