\section{Firewall Correctness}
\label{sec:correctness}

The verification process is important and if the firewalls correctness can't be assured, the system behind the firewall can be endangered. The verification process for the final firewall configuration was performed in two steps. The first step was human inspection, we looked through the rules and their order and assured us of the correctness. The second step was performed by launching various attacks towards the system hosting the firewall, from a computer on the same local network. The attacks carried out were among others different kinds of TCP port scans and ICMP pings.

The first step was as mentioned above, human inspection. As explained in section~\ref{sec:config} we took care in which order the rules were placed. The followed concept of first having a number of rules, the denying rules, that filtered away unwanted packages that would otherwise - possibly - be accepted by the following number of rules, the allowing rules, that accepts wanted packages. As also exlained in section~\ref{sec:config}, if a packege neither is matched with the two rule sets the packege will face the default policy, which was set to DROP the package. The denying rules span from line 4 to 11 on the input chain and from line 31 to 46 on the output chain, see Listening \ref{list:final_firewall_config}. Internally in each chain it does not matter in which order they are placed, but there is one exception. The lines 10 and 11 has to be placed last of the denying rules in the input chain due to that those two lines regarding the ICMP ping filtering are composed by a dropping and an accepting rule. Because it holds a accepting rule they has to be placed last so that all unwanted traffic first are filtered away before it may reach the ICMP filtering rules. Regarding the applying rules on line By using the explained conscept and the ordering explained inside each two rule sets we belive that the ordering of the rules are correct.

\inlinetodo{Explain which tool you used and how it helped you in verifying your firewall configuration. Elaborate on why the firewall is correctly configured and does what it should do. E.g., by trying the command XXX, we found that there are only YYY number of packets returned when pinging the host. Thus, the ping protection (rule Z) is working. \\
Also answer:\\
-- Why is the order of your firewall rules correct and makes sense?\\
-- Is your configuration stateful?}


