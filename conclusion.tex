\section{Conclusion} 
\label{sec:conclusion}
A script that configures the linux firewall using the tool \texttt{iptables} have been made. The firewall have been configured in a statefull way and to only allow NFS, SSH, portmapper, and apache through from the outside. It have also been configured to block ping flooding, mallformed TCP packets, and some level of IP address spoofing.

The purpose of this \lab{} was to gain an basic understanding of the concept of network firewalls and even though some rules were not verified, as discussed in Section \ref{sec:discussion}, and that we had some problems it feels like the purpose have been met. We can now afterwards say that a firewall that previously felt like something abstract and quite hard to understand have now become manageable in terms of understanding. Also the fact that it is hard to configure a perfect firewall in terms of security and usability have become obvious, a tradeoff almost always have to be made and it is easy to miss something, especially when configuring a firewall for example huge networks with many hosts, servers and firewalls. Concerning the tool \texttt{iptables} that have been used to configure the Linux firewall we have seen that it is a incredibly powerful and versatile tool that with the right configuration can protect from the most threats. 